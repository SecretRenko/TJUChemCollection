\documentclass[lang=cn,newtx,10pt,scheme=chinese]{elegantbook}

\xeCJKsetup{AutoFakeBold=true}%%解决中文不能加粗的问题
\setCJKmainfont[BoldFont={FZHei-B01},ItalicFont={FZKai-Z03}]{FZShuSong-Z01}
\setCJKsansfont{FZHei-B01}
\setCJKfamilyfont{zhsong}{FZShuSong-Z01}
\setCJKfamilyfont{zhhei}{FZHei-B01}
\setCJKfamilyfont{zhkai}{FZKai-Z03}
\setCJKfamilyfont{zhfs}{FZFangSong-Z02}
\setCJKfamilyfont{zhli}{FZLiShu-S01}


\usepackage{geometry}
\usepackage{graphicx}
\usepackage{epstopdf}
\usepackage{chemformula}
\usepackage{mhchem}

\usepackage{caption}
\usepackage{subcaption}

%计数器
\newcounter{MajorCount}
\newcounter{Section}[MajorCount]
\newcounter{MinorCount}[Section]


%%大题环境
\newenvironment{NT}[1]{\stepcounter{MajorCount} \noindent \textbf{第\theMajorCount 题~#1} \vspace{0.5em}}{\vspace{1em}}

%\newcommand{\NT}[1]{\stepcounter{MajorCount} \noindent \textbf{第\theMajorCount 题~#1} \vspace{0.5em}}

%%分题号
\newcommand{\Sec}[1]{\vspace{0.2em}\stepcounter{Section}\noindent\textbf{\theMajorCount-\theSection~\fangsong{#1}}\vspace{0.2em}}

%%小题号
\newcommand{\Se}{\stepcounter{MinorCount}\noindent \textbf{\theMajorCount-\theSection-\theMinorCount~}}

%%选做题
\newcommand{\Sel}{\stepcounter{MinorCount}\noindent \textbf{\theMajorCount-\theSection-\theMinorCount~\textbf{\kaishu (选做)}~}}

%%%%%%%%%%%%%%%%%%%%%

\title{ElegantBook:优美的 \LaTeX{} 书籍模板}
\subtitle{Elegant\LaTeX{} 经典之作}

\author{Ethan Deng \& Liam Huang \& syvshc \& sikouhjw \& Osbert Wang}
\institute{Elegant\LaTeX{} Program}
\date{2025/12/14}
\version{0.1}
\bioinfo{自定义}{信息}

\extrainfo{注意:本模板自 2023 年 1 月 1 日开始,不再更新和维护!}

\setcounter{tocdepth}{3}

\logo{logo-blue.png}
\cover{cover.jpg}

% 本文档命令
\usepackage{array}
\newcommand{\ccr}[1]{\makecell{{\color{#1}\rule{1cm}{1cm}}}}

% 修改标题页的橙色带
\definecolor{customcolor}{RGB}{32,178,170}
\colorlet{coverlinecolor}{customcolor}
\usepackage{cprotect}

\graphicspath{{Reac/}{image/}{figure/}}%设置图片目录

\begin{document}



\chapter*{25-26学年第一学期高等有机化学期末试题(回忆版)}
\vspace*{-1em}
\begin{quotation}
	\kaishu
	
	本资料旨在让备考的同学们了解考试难度,心里有个底。不要指望着写完这套卷子就能稳了,还请认真复习。因难度较低,故不提供答案,自行查书即可。
	
	
	回忆版试题仅供参考,不保证与真题完全一致。\textcolor{red}{整理者不对本文档的读者的成绩负责!!请勿拿着本文档去问老师问题,或以其他任何方式将本习题泄露给老师或学院。}
	
	本课程序号:00410
	
	考试时间:第14周~~2025.12.14~8:00~~\textasciitilde~10:00
	
	试题最后更新时间:2025.12.14
	
\end{quotation}

\vspace*{-1em}

\section*{一、解释题}
\subsection*{1.一个甾体手性碳的构型判断}

标出分子中所有手性碳的相对构型。
\begin{figure}[ht]
	\centering
	\includegraphics[width=0.4\textwidth]{1}
\end{figure}
\vspace*{-0.5em}

\subsection*{2.超共轭效应}
试从超共轭的角度出发,论证为何稳定性顺序为:
\begin{enumerate}
	\item $Me_3C^+~>~MeH_2C^+~>~H_3C^+$
	\item $Me_3C\cdot~~>~MeH_2C\cdot~~>~H_3C\cdot$
\end{enumerate}
\vspace{0.5em}

\subsection*{3.反应活性}
如下图所示,两结构相似的化合物在相同条件下反应性截然不同。试解释原因。

\begin{figure}[ht]
	\centering
	\includegraphics[width=0.35\textwidth]{2}
\end{figure}

\newpage

\section*{二、异丙苯法制苯酚与Nylon~6的合成}

写出括号中的结构,并画出Beckmann重排和第三步的详细机理。

\begin{figure}[ht]
	\centering
	\vspace{1em}
	\includegraphics[width=0.95\textwidth]{3}
\end{figure}

%\newpage

\section*{三、基础反应考察}
\begin{enumerate}
	\item 写出括号中物质的结构。其中$R$为大位阻烷基\\
		\begin{minipage}[h]{1\textwidth}
			\vspace{1em}
			\centering
			\includegraphics[width=0.5\textwidth]{3-1}
		\end{minipage}
   		\vspace{1em}
	
	\item 写出括号中物质的结构\\
		\begin{minipage}[h]{1\textwidth}
   		 	\vspace{1em}	
			\centering
			\includegraphics[width=0.7\textwidth]{3-2}
		\end{minipage}
		\vspace{1em}
	
	\item 写出括号中物质的结构\\
		\begin{minipage}[h]{1\textwidth}
			\vspace{1em}	
			\centering
			\includegraphics[width=0.6\textwidth]{3-3}
		\end{minipage}
		\vspace{1em}
	
	\item 按照稳定性顺序排序下列物质\\
		\begin{minipage}[h]{1\textwidth}
			\vspace{1em}	
			\centering
			\includegraphics[width=0.7\textwidth]{3-4}
		\end{minipage}
		\vspace{1em}

	%\newpage
	
	\item 写出括号中物质的结构\\
		\begin{minipage}[h]{1\textwidth}
			\vspace{1em}	
			\centering
			\includegraphics[width=0.7\textwidth]{3-5}
		\end{minipage}
		\vspace{1em}
		
	\item 写出括号中物质的结构\\
		\begin{minipage}[h]{1\textwidth}
			\vspace{1em}	
			\centering
			\includegraphics[width=0.48\textwidth]{3-6}
		\end{minipage}
		\vspace{1em}
		
	\item 写出括号中物质的结构\\
		\begin{minipage}[h]{1\textwidth}
			\vspace{1em}	
			\centering
			\includegraphics[width=0.9\textwidth]{3-7}
		\end{minipage}
		\vspace{1em}
		
	\item 写出括号中物质的结构\\
		\begin{minipage}[h]{1\textwidth}
			\vspace{1em}	
			\centering
			\includegraphics[width=0.45\textwidth]{3-8}
		\end{minipage}
		\vspace{1em}

\end{enumerate}

%\newpage

\section*{四、周环反应}

纯净的光甾醇久置后纯度会下降,体系中出现了麦角固醇、焦钙化甾醇、异焦钙化甾醇。试解释原因。

\kaishu(提示:经过了预钙化甾醇中间体) 

\begin{minipage}[h]{1\textwidth}
	\vspace{1em}	
	\centering
	\includegraphics[width=0.3\textwidth]{4}
	\vspace{1em}
	\captionof*{figure}{\fangsong 光甾醇的结构,其余结构略。}
\end{minipage}

\vspace{10em}

\begin{postulate*}
	\vspace{0.2em}
	\hspace{2em}本回忆版试题由24级应用化学(拔尖班)制作。感谢张润森和张永祺同学帮助补充完善了第三大题,感谢~Elegant\LaTeX{} Program。
	
	\hspace{2em}再次强调,请勿将本习题泄露给老师或学院,否则可能导致你自己的期末题目变得不可预测。另外,禁止任何形式的商业用途。
	
	\hspace{2em}本项目遵循~\href{https://creativecommons.org/licenses/by-nc-sa/4.0/}{CC BY-NC-SA 4.0}~开源许可协议。原始$.tex$文件已开源至~\href{https://github.com/SecretRenko/TJUChemCollection}{Github}。如本试题中有不准确的地方,欢迎各位补充指正。
\end{postulate*}


\end{document}